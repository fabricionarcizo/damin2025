%
% MIT License
%
% Copyright (c) 2025 Fabricio Batista Narcizo
%
% Permission is hereby granted, free of charge, to any person obtaining a copy
% of this software and associated documentation files (the "Software"), to deal
% in the Software without restriction, including without limitation the rights
% to use, copy, modify, merge, publish, distribute, sublicense, and/or sell
% copies of the Software, and to permit persons to whom the Software is
% furnished to do so, subject to the following conditions:
%
% The above copyright notice and this permission notice shall be included in
% all copies or substantial portions of the Software.
%
% THE SOFTWARE IS PROVIDED "AS IS", WITHOUT WARRANTY OF ANY KIND, EXPRESS OR
% IMPLIED, INCLUDING BUT NOT LIMITED TO THE WARRANTIES OF MERCHANTABILITY,
% FITNESS FOR A PARTICULAR PURPOSE AND NONINFRINGEMENT. IN NO EVENT SHALL THE
% AUTHORS OR COPYRIGHT HOLDERS BE LIABLE FOR ANY CLAIM, DAMAGES OR OTHER
% LIABILITY, WHETHER IN AN ACTION OF CONTRACT, TORT OR OTHERWISE, ARISING FROM,
% OUT OF OR IN CONNECTION WITH THE SOFTWARE OR THE USE OR OTHER DEALINGS IN THE
% SOFTWARE.
%

% This example demonstrates how to include a static plot exported as PGF and an
% animated plot.
\documentclass{article}

% Used LaTeX Packages.
\usepackage{graphicx}
\usepackage{animate} % core package
\usepackage{pgf} % PGF loader

% Ensure compatibility with math fonts in LaTeX documents.
\makeatletter
\@ifundefined{mathdefault}{\newcommand{\mathdefault}[1]{#1}}{}
\makeatother

\begin{document}
    Figure~\ref{fig:sine-pgf} shows a sine wave plot generated in Python using Matplotlib and exported as a PGF file. The PGF format allows for seamless integration with LaTeX documents, ensuring that fonts and styles match the rest of the document.
    \begin{figure}[h]
        \centering
        % scale with column width:
        \resizebox{\linewidth}{!}{\input{sine.pgf}}
        \caption{Sine wave exported from Matplotlib as PGF.}
        \label{fig:sine-pgf}
    \end{figure}

    \newpage
    Figure~\ref{fig:sine-anim} shows an animated sine wave with a phase shift, created by generating multiple frames in Python and compiling them into an animation using the \texttt{animate} package in LaTeX.
    \begin{figure}[h]
        \centering
        \animategraphics[
            controls,loop,autoplay,        % UI + behavior
            poster=first,                  % still frame when not playing
            width=\linewidth,              % scale widget
            type=pdf                       % force the extension if needed
        ]{12}{frames/sine_}{000}{120}      % 12 fps, basename, first..last
        \caption{Animated sine with phase shift.}
        \label{fig:sine-anim}
    \end{figure}
\end{document}
